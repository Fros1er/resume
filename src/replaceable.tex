\newpage

\section{\faGraduationCap\  可添加的教育背景}
\datedsubsection{\textbf{南方科技大学},深圳}{2020 -- 2024}
本科,计算机科学与技术。专业课程绩点:3.79/4.00
\begin{itemize}[parsep=0.5ex]
  \item A 及以上课程:操作系统、计算机网络、软件工程、数字逻辑、编译原理、计算机安全、组成原理。
\end{itemize}

\section{\faBook\ 可添加的专业技能}
\begin{itemize}[parsep=0.5ex]
    \item 查阅过RISC-V ISA、debug标准、Vector扩展的文档。对RISC-V有一定的了解。
\end{itemize}

\section{\faCogs\ 可添加的项目经历}


\begin{onehalfspacing}
    \datedsubsection{\textbf{基于 git 的在线版本控制和开发协作平台}}{}
  课程项目,合作实现了在网页上的 git 仓库访问、commit,diff 和历史版本查看、issue,pull request、用户信息查看、CI/CD 配置等功能。项目使用 vue3+springboot+postgresql 开发。个人在项目中承担主要开发人员的角色:负责主要功能实现、开发规范制定、技术选型、代码审计和任务分配;负责前端部分的开发,采用 typescript+vue3+pinia+element3+unocss等主流技术;使用 vue-router 导航守卫在路由切换时防止多次请求。使用 axios 拦截器进行错误提示和 404 页跳转。
  \end{onehalfspacing}

\begin{onehalfspacing}
  \datedsubsection{\textbf{一种c风格简易编程语言的编译器}}{}
  课程项目。针对一种c风格简易编程语言,使用C++实现了编译器的前端。项目使用flex和bison进行词法分析和语法分析,并生成AST。项目随后利用AST生成中间代码表示(四地址码),进行不可达代码消除和限定条件下的复制传播优化和条件跳转简化。
%   项目介绍:针对一种 c 风格简易编程语言,使用 c++ 配合 flex 和 bison 实现了编译器中的词法分析、
% 语法分析、语义分析和中间代码生成几部分。
% 项目职责:
% • 构建符合 flex 和 bison 规范的规则和产生式,并用上述工具构建语法树。
% • 在语义分析中,使用节点中包含哈希表的树在不同作用域中维护符号表并进行错误检查;使用
% variant 代替 union 在 AST 节点中维护不同类型的数据;应用简单的 RTTI 思想记录 AST 节点类型。
% • 项目使用智能指针,移动语义等特性减少拷贝,防止内存泄漏。
% -----------
% 项目介绍:针对一种 c 风格简易编程语言,使用 c++ 配合 flex 和 bison 实现了编译器的前端。
%在语义分析中,使用节点中包含哈希表的树在不同作用域中维护符号表并进行错误检查;使用
% variant 代替 union 在 AST 节点中维护不同类型的数据;应用简单的 RTTI 思想记录 AST 节点类型。
% • 项目使用智能指针,移动语义等特性减少拷贝,防止内存泄漏。

\end{onehalfspacing}

% \begin{onehalfspacing}
%   \datedsubsection{\textbf{其他RISC-V相关项目}}{}
%   在某比赛中使用RISC-V Vector扩展加速GMSSL运算。在学校项目中使用RISC-V debug的Trigger配合GDB stub实现GDB远程调试裸机程序。
% \end{onehalfspacing}

\section{\faHeartO 可添加的获奖情况}
\begin{onehalfspacing}
\datedsubsection{RoboMaster 2021 机甲大师超级对抗赛全国赛,\textbf{一等奖}}{2021 年 7 月}
在队伍中主要负责嵌入式系统部分。负责机器人的基础功能实现、调试,并利用 ttl 串口实现单片
机和运行在 linux 系统下的 c++ 程序通讯。
\end{onehalfspacing}
\begin{onehalfspacing}
    \datedsubsection{第六届“强网杯”全国网络安全挑战赛线上赛,\textbf{强网先锋}}{2022 年 7 月
    }
    在比赛中与队友合作解出一道 web。个人了解一些web漏洞利用和二进制逆向的相关知识。
\end{onehalfspacing}

\end{document}