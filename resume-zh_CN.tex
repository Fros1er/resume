% !TEX TS-program = xelatex
% !TEX encoding = UTF-8 Unicode
% !Mode:: "TeX:UTF-8"

\documentclass{resume}
\usepackage{zh_CN-Adobefonts_external} % Simplified Chinese Support using external fonts (./fonts/zh_CN-Adobe/)
%\usepackage{zh_CN-Adobefonts_internal} % Simplified Chinese Support using system fonts
\usepackage{linespacing_fix} % disable extra space before next section
\usepackage{cite}
\usepackage{graphicx}

\begin{document}
\pagenumbering{gobble} % suppress displaying page number

\name{徐延楷}

\basicInfo{
  \email{12011525@mail.sustech.edu.cn} \textperiodcentered\
  \phone{(+86) 189-5322-0690} \textperiodcentered\
  \github[Fros1er]{https://github.com/Fros1er}}

\section{\faGraduationCap\  教育背景}
\datedsubsection{\textbf{南方科技大学},深圳}{2020 -- 2024}
本科,计算机科学与技术。

\section{\faBook\ 专业技能}
\begin{itemize}[parsep=0.5ex]
  \item 熟悉C/C++开发,熟悉C++17及常用stl,能够使用gdb、make、cmake、vcpkg。
  \item 工作电脑是ubuntu,较少遇到配环境问题。熟练使用git。较为熟练的使用java、rust、python、js。
  \item CET6 586/710,熟练阅读英文技术文档。能够在github上使用英文进行交流。
  \item 查阅过RISC-V ISA、debug标准、Vector扩展的文档。对RISC-V有一定的了解。
  \item 有独立测试、调试、分析查错的能力,懂得如何提问。
\end{itemize}

\newcommand{\customtitle}[2]{\parbox[t]{#1}{\raggedright #2}}

\section{\faUsers\ 实习经历}
\begin{onehalfspacing}
  \datedrolesubsection{ApeCloud}{数据库系统开发实习}{2023年7月 -- 2023年9月}
  在fork自阿里PolarDB-X的数据库上进行开发。工作包括:在数据库中优化Decimal相关的计算和存储、修复SQL命令执行相关的bug、添加自动化测试并配置CI/CD、重构已有代码。在工作任务之外开发了一个自动化脚本,用于快速进行常用配置和手动测试,被广泛使用并提升了项目组工作效率。
\end{onehalfspacing}

\begin{onehalfspacing}
  \datedrolesubsection{深圳国际量子研究院}{软件开发}{2022年11月 -- 2023年5月}
  使用C++精简实验室测控用RPC系统scalabrad。系统类似一个RPC系统的registry,支持节点的注册、查找、服务调用和消息通知。系统使用cmake和vcpkg构建到不同OS,使用boost::asio的coroutine维护子节点tcp连接和收发消息,使用boost::qi将LabRad协议文本解析为AST,通过模板实现仅使用一次内存分配构建固定格式的数据包。精简后的系统改善了启动时间和资源占用,已被实验室接收。
\end{onehalfspacing}

\section{\faCogs\ 项目经历}
\begin{onehalfspacing}
  \datedsubsection{\textbf{Nemu-Rust}}{}
  使用Rust从头实现南京大学《计算机系统基础》课程包含的RISC-V ISA模拟器。项目包含RV64I指令的模拟、基于sdl的外设模拟、简易调试器以及qemu difftest。模拟器性能达到native的4.7\%,可以运行红白机游戏。
\end{onehalfspacing}

\begin{onehalfspacing}
  \datedsubsection{\textbf{一种c风格简易编程语言的编译器}}{}
  课程项目。针对一种c风格简易编程语言,使用C++实现了编译器的前端。项目使用flex和bison进行词法分析和语法分析,并生成AST。项目随后利用AST生成中间代码表示(四地址码),进行不可达代码消除和限定条件下的复制传播优化和条件跳转简化。
\end{onehalfspacing}

\begin{onehalfspacing}
  \datedsubsection{\textbf{其他RISC-V相关项目}}{}
  在某比赛中使用RISC-V Vector扩展加速GMSSL运算。在学校项目中使用RISC-V debug的Trigger配合GDB stub实现GDB远程调试裸机程序。
\end{onehalfspacing}



\section{\faGithub\ 社区参与}
\subsection{fmt~PR\#3293}
为c++字符串格式化库fmt添加在windows下通过utf-16路径打开文件的支持。已被开发者合并。

\subsection{AmazeFileManager~PR\#3226}
修复了开源安卓文件管理器AmazeFileManager中文件列表为空指针、分享文件来源判断错误两个bug。已被开发者合并。

\end{document}
