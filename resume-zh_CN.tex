% !TEX TS-program = xelatex
% !TEX encoding = UTF-8 Unicode
% !Mode:: "TeX:UTF-8"

\documentclass{resume}
\usepackage{zh_CN-Adobefonts_external} % Simplified Chinese Support using external fonts (./fonts/zh_CN-Adobe/)
%\usepackage{zh_CN-Adobefonts_internal} % Simplified Chinese Support using system fonts
\usepackage{linespacing_fix} % disable extra space before next section
\usepackage{cite}
\usepackage{graphicx}
\usepackage{hyperref}
\hypersetup{
  colorlinks=false,
  linkcolor=blue,
  filecolor=magenta,      
  urlcolor=blue,
  pdftitle={Overleaf Example},
  pdfpagemode=FullScreen,
}

\urlstyle{same}

\begin{document}
\pagenumbering{gobble} % suppress displaying page number

\name{这是名字}

\basicInfo{
  \email{12011525@mail.sustech.edu.cn} \textperiodcentered\
  \phone{(+86) 189-5322-0690} \textperiodcentered\
  \github[Fros1er]{https://github.com/Fros1er}}

\section{\faGraduationCap\  教育背景}
\datedsubsection{\textbf{南方科技大学},深圳}{}
\datedline{硕士,计算机科学与技术。}{2024 -- 预期2027毕业}
\datedline{本科,计算机科学与技术。专业课程绩点:3.79/4.00}{2020 -- 2024}


\section{\faBook\ 专业技能}
\begin{itemize}[parsep=0.5ex]
  \item 熟悉C/C++开发,熟悉面向对象编程思想及常用设计模式,熟悉C++17及常用stl,能够使用gdb、make、cmake、vcpkg。
  \item 熟悉 Socket 网络编程,熟悉 TCP/IP,HTTP 等网络协议, 了解 select、poll、epoll 多路 IO 复用模型。
  \item 工作电脑使用Ubuntu,较少遇到配环境问题。熟悉 Linux 下常用命令,熟悉git使用。
  \item 熟悉java、rust、python、js、数据库的使用。有独立测试、调试、分析查错的能力,懂得如何提问。
  \item CET6 586/710,熟练阅读英文技术文档。能够在github上使用英文进行交流。
\end{itemize}

\section{\faUsers\ 实习经历}
\begin{onehalfspacing}
  \datedrolesubsection{ApeCloud}{数据库系统开发实习}{2023年7月 -- 2023年9月}
  在fork自阿里PolarDB-X的数据库上进行开发。工作包括:在数据库中优化Decimal相关的计算和存储、修复SQL命令执行相关的bug、添加自动化测试并配置CI/CD、重构已有代码。在工作任务之外开发了自动化脚本,用于快速进行常用配置和测试,被项目组内广泛使用,提升了工作效率。
\end{onehalfspacing}

\begin{onehalfspacing}
  \datedrolesubsection{深圳国际量子研究院}{软件开发}{2022年11月 -- 2023年5月}
  使用C++重构并精简实验室测控用RPC系统scalabrad。系统类似一个RPC系统的registry,支持节点的注册、查找、服务调用和消息通知。系统使用cmake和vcpkg构建到不同OS;使用boost::asio的coroutine维护子节点tcp连接并收发消息;使用boost::qi解析LabRad协议数据包文本;对于固定格式数据包的构建,使用模板优化到在一次内存分配下构建;支持网络错误重试,支持服务可用性改变时对节点的广播。精简后的系统改善了启动时间和资源占用,已被实验室接收。
\end{onehalfspacing}

\section{\faCogs\ 项目经历}

\begin{onehalfspacing}
  \datedsubsection{\textbf{轻量级 c++ 无栈协程调度器与 TCP 服务器}}{\url{https://github.com/Fros1er/coro-server/}}
  使用 c++20 无栈协程特性,借鉴 golang 的调度算法实现了简易的协程调度器,并基于该调度器和 epoll 实现了 TCP 服务器。项目的调度算法实现了全局运行队列、全局等待队列和局部运行队列,支持将协程分发至不同工作线程,支持工作窃取算法;项目使用 mutex 保证资源安全,使用条件变量实现无任务时的线程挂起;项目实现了基于优先队列的协程挂起,基于原子操作的协程内互斥锁,和epoll 阻塞模式在协程内的封装;项目使用 epoll 的 ET 模式监听事件,使用基于协程的 Reactor 并发模型进行事件处理。
\end{onehalfspacing}

\begin{onehalfspacing}
  \datedsubsection{\textbf{Nemu-Rust (Rust实现的RISC-V 模拟器)}}{\url{https://github.com/Fros1er/Nemu-rust/}}
  使用 Rust 从头实现南京大学《计算机系统基础》课程包含的 RISC-V ISA 模拟器。项目包含对 rv64ima 指令集、sv39 分页机制和权限级别的模拟、基于 sdl 和 tokio 的外设及中断模拟、简易调试器以及 qemu difftest。模拟器性能达到 native 的 4.7\%,可以运行仙剑奇侠传、opensbi 和教学用 OS。

\end{onehalfspacing}



\section{\faGithub\ 社区参与}
\begin{itemize}[parsep=0.5ex]
  \item {\large \textbf{llvm-project:}} 3PRs, 2Issues。在llvm后端中添加对特定代码的RISC-V V扩展自动向量化支持。
  \item {\large \textbf{fmt:}} PR\#3293。添加在windows下通过utf-16路径打开文件的支持。
  \item {\large \textbf{AmazeFileManager:}} PR\#3226。修复文件列表为空指针、分享文件来源判断错误两个bug。
\end{itemize}


% ---------------------------------------------

% \newpage

\section{这一页是可选的,可能根据岗位和上面的换一下}

\vspace{0.5cm}

\section{而且没有优化过写作}

\vspace{1cm}

\section{\faGraduationCap\  可添加的教育背景}
\datedsubsection{\textbf{南方科技大学},深圳}{2020 -- 2024}
本科,计算机科学与技术。专业课程绩点:3.79/4.00
\begin{itemize}[parsep=0.5ex]
  \item A 及以上课程:操作系统、计算机网络、软件工程、数字逻辑、编译原理、计算机安全、组成原理。
\end{itemize}

\section{\faBook\ 可添加的专业技能}
\begin{itemize}[parsep=0.5ex]
    \item 查阅过RISC-V ISA、debug标准、Vector扩展的文档。对RISC-V有一定的了解。
\end{itemize}

\section{\faCogs\ 可添加的项目经历}


\begin{spacing}{1.125}
    \datedsubsection{\textbf{基于 git 的在线版本控制和开发协作平台}}{}
  课程项目,合作实现了在网页上的 git 仓库访问、commit,diff 和历史版本查看、issue,pull request、用户信息查看、CI/CD 配置等功能。项目使用 vue3+springboot+postgresql 开发。个人在项目中承担主要开发人员的角色:负责主要功能实现、开发规范制定、技术选型、代码审计和任务分配;负责前端部分的开发,采用 typescript+vue3+pinia+element3+unocss等主流技术;使用 vue-router 导航守卫在路由切换时防止多次请求。使用 axios 拦截器进行错误提示和 404 页跳转。
  \end{spacing}

% \begin{onehalfspacing}
%   \datedsubsection{\textbf{一种c风格简易编程语言的编译器}}{}
%   课程项目。针对一种c风格简易编程语言,使用C++实现了编译器的前端。项目使用flex和bison进行词法分析和语法分析,并生成AST。项目随后利用AST生成中间代码表示(四地址码),进行不可达代码消除和限定条件下的复制传播优化和条件跳转简化。
%   项目介绍:针对一种 c 风格简易编程语言,使用 c++ 配合 flex 和 bison 实现了编译器中的词法分析、
% 语法分析、语义分析和中间代码生成几部分。
% 项目职责:
% • 构建符合 flex 和 bison 规范的规则和产生式,并用上述工具构建语法树。
% • 在语义分析中,使用节点中包含哈希表的树在不同作用域中维护符号表并进行错误检查;使用
% variant 代替 union 在 AST 节点中维护不同类型的数据;应用简单的 RTTI 思想记录 AST 节点类型。
% • 项目使用智能指针,移动语义等特性减少拷贝,防止内存泄漏。
% -----------
% 项目介绍:针对一种 c 风格简易编程语言,使用 c++ 配合 flex 和 bison 实现了编译器的前端。
%在语义分析中,使用节点中包含哈希表的树在不同作用域中维护符号表并进行错误检查;使用
% variant 代替 union 在 AST 节点中维护不同类型的数据;应用简单的 RTTI 思想记录 AST 节点类型。
% • 项目使用智能指针,移动语义等特性减少拷贝,防止内存泄漏。

% \end{onehalfspacing}

% \begin{onehalfspacing}
%   \datedsubsection{\textbf{其他RISC-V相关项目}}{}
%   在某比赛中使用RISC-V Vector扩展加速GMSSL运算。在学校项目中使用RISC-V debug的Trigger配合GDB stub实现GDB远程调试裸机程序。
% \end{onehalfspacing}

\section{\faHeartO 可添加的获奖情况}
\begin{spacing}{1.125}
\datedsubsection{RoboMaster 2021 机甲大师超级对抗赛全国赛,\textbf{一等奖}}{2021 年 7 月}
在队伍中主要负责嵌入式系统部分。负责机器人的基础功能实现、调试,并利用 ttl 串口实现单片
机和运行在 linux 系统下的 c++ 程序通讯。
\end{spacing}
\begin{spacing}{1.125}
    \datedsubsection{第六届“强网杯”全国网络安全挑战赛线上赛,\textbf{强网先锋}}{2022 年 7 月
    }
\end{spacing}

\end{document}



\end{document}
