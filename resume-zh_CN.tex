% !TEX TS-program = xelatex
% !TEX encoding = UTF-8 Unicode
% !Mode:: "TeX:UTF-8"

\documentclass{resume}
\usepackage{zh_CN-Adobefonts_external} % Simplified Chinese Support using external fonts (./fonts/zh_CN-Adobe/)
%\usepackage{zh_CN-Adobefonts_internal} % Simplified Chinese Support using system fonts
\usepackage{linespacing_fix} % disable extra space before next section
\usepackage{cite}
\usepackage{graphicx}
\usepackage{hyperref}
\usepackage{setspace}
\usepackage{titlesec}

\titlespacing*{\subsection} {0pt}{-1ex}{0ex}
\titlespacing*{\section} {0pt}{-0.5ex}{1.5ex}

\hypersetup{
  colorlinks=false,
  linkcolor=blue,
  filecolor=magenta,      
  urlcolor=blue,
  pdftitle={Overleaf Example},
  pdfpagemode=FullScreen,
}

\urlstyle{same}

\begin{document}
\pagenumbering{gobble} % suppress displaying page number

% \name{徐延楷}
\name{这是名字}

\basicInfo{
  % \email{12011525@mail.sustech.edu.cn} \textperiodcentered\
  % \phone{(+86) 189-5322-0690} \textperiodcentered\
  % \github[Fros1er]{https://github.com/Fros1er}}
  \email{school\_mail@mail.sustech.edu.cn} \textperiodcentered\
  \phone{(+86) 这是电话} \textperiodcentered\
  \github[Fros1er]{https://github.com/Fros1er}}

\section{\faGraduationCap\  教育背景}
\datedsubsection{\textbf{南方科技大学},深圳}{}
\begin{spacing}{1.125}
\datedline{硕士,计算机科学与技术。}{2024 -- 预期2027毕业}
\datedline{本科,计算机科学与技术。专业课程绩点:3.79/4.00}{2020 -- 2024}
\end{spacing}

\section{\faBook\ 专业技能}
\begin{itemize}[parsep=0.5ex]
  \item 熟悉C/C++开发,C++17及常用stl。能够使用gdb、make、cmake、vcpkg进行调试与构建。
  \item 熟悉面向对象编程、设计模式。了解Linux系统编程及网络编程,如IO多路复用模型,协程等技术。
  \item 熟悉Java、Python、JS、Rust、SQL。熟悉git使用,了解Linux下常用命令。
  \item 能快速学习并切入大型项目开发。具备独立调试、分析、排错的能力,懂得如何提问。
  \item CET6 586/710,能流畅阅读英文技术文档。在GitHub参与过大型项目的开源贡献,能使用英文交流。
\end{itemize}

\section{\faUsers\ 实习经历}

\datedrolesubsection{蚂蚁密算}{研究型实习}{2024年11月 -- 至今}
\begin{spacing}{1.125}
  合作进行\textbf{xPU TEE}相关项目的论文写作,并协助在arm设备上配置基于TPM的u-boot+linux kernel可信启动。\textbf{(TODO)这个刚开始没多久,不知道写啥}
\end{spacing}

% \datedrolesubsection{PLCT实验室}{}{2024年7月 -- 2024年10月}
% \begin{spacing}{1.125}
% 搞LLVM,给RISC-V V自动向量化交了三个pr,合了。不想写这个实习。经历太杂了,列出来都想笑。
% \end{spacing}

\datedrolesubsection{ApeCloud}{数据库系统开发实习}{2023年7月 -- 2023年9月}
\begin{spacing}{1.125}
  参与\textbf{基于PolarDB-X fork和Amazon S3}的一款\textbf{OLAP数据库}开发。贡献包括:以层次化的方式重新组织\textbf{对象存储目录结构},利用\textbf{S3 prefix}简化管理;修复\textbf{SQL执行、数据文件管理}相关bug;参考现有方案,\textbf{优化DECIMAL存储},将能够由\textbf{int128存储的数值进行压缩},减少存储开销;在工作任务之外主动\textbf{重构代码}以提高可维护性,并开发\textbf{自动化脚本}加速配置和测试流程,被项目组内广泛使用以提高效率。
\end{spacing}

\datedrolesubsection{深圳国际量子研究院}{软件开发}{2022年11月 -- 2023年5月}
\begin{spacing}{1.125}
  \textbf{独立}使用C++重构并裁剪实验室测控系统scalabrad。该系统类似\textbf{中心化的RPC框架},用于控制和读取实验设备。系统使用\textbf{Boost.Asio}以\textbf{单线程协程}模型实现TCP长连接,减少同步锁开销,简化代码逻辑;结合\textbf{源码分析和Wireshark抓包}逆向解析协议,完整复刻行为;基于\textbf{Boost.Qi}解析复杂协议文本结构;使用\textbf{变长模板在编译期计算数据包大小},提高构建效率。系统\textbf{服务调用加速13\%,内存占用仅为1\%}。
\end{spacing}

\section{\faCogs\ 项目经历}


\datedsubsection{\textbf{轻量级C++无栈协程调度器与TCP服务器}}{\url{https://github.com/Fros1er/coro-server/}}
\begin{spacing}{1.125}
  基于\textbf{C++20无栈协程},借鉴\textbf{Golang协程调度算法}实现轻量级协程调度器。项目支持\textbf{跨线程协程调度},实现了\textbf{全局/局部运行队列、工作窃取},减少线程负载不均问题;提供\textbf{基于优先队列的协程调度,基于原子操作的协程内互斥锁,以及epoll阻塞模式的协程封装};基于该协程调度器实现了TCP服务器,采用\textbf{epoll ET模式,并结合Reactor并发模型}进行事件处理。\textbf{TODO:添加benchmark}
\end{spacing}

\datedsubsection{\textbf{Nemu-Rust (Rust实现的RISC-V 模拟器)}}{\url{https://github.com/Fros1er/Nemu-rust/}}
\begin{spacing}{1.125}
  使用Rust实现RISC-V模拟器,支持\textbf{rv64ima 指令集、sv39分页、特权级、外设、中断}。使用\textbf{perf、火焰图、AMD uProf}进行性能调优,实现与qemu的\textbf{difftest}进行正确性验证。裸机环境下microbench性能达 \textbf{native CPU的0.75\%},可运行\textbf{仙剑奇侠传,完整启动opensbi+Linux(可在busybox shell运行CoreMark)}。

\end{spacing}



\section{\faGithub\ 社区参与}
\begin{spacing}{1.125}
\begin{itemize}[parsep=0.5ex]
  \item {\large \textbf{llvm-project:}} 3PRs, 2Issues。PR\#95563修复trunc(srl) 误优化,修复RISC-V V 扩展下truncate指令生成冗余;PR\#100749放宽binop(splat, splat)的标量化条件,使非合法类型得以优化,影响十几个test case,解法获maintainer认可。Issue\#104596通过benchmark发现特定函数无法向量化,并定位原因。
  \item {\large \textbf{fmt:}} PR\#3293通过新增API和字符编码转换,使fmt::file支持\textbf{Windows UTF-16文件路径}。
  \item {\large \textbf{AmazeFileManager:}} PR\#3226修复文件列表为空指针、分享文件来源判断错误两个bug。
\end{itemize}
\end{spacing}

% ---------------------------------------------

\newpage

\section{这一页是可选的,可能根据岗位和上面的换一下}

\vspace{0.5cm}

\section{而且没有优化过写作}

\vspace{1cm}

\section{\faGraduationCap\  可添加的教育背景}
\datedsubsection{\textbf{南方科技大学},深圳}{2020 -- 2024}
本科,计算机科学与技术。专业课程绩点:3.79/4.00
\begin{itemize}[parsep=0.5ex]
  \item A 及以上课程:操作系统、计算机网络、软件工程、数字逻辑、编译原理、计算机安全、组成原理。
\end{itemize}

\section{\faBook\ 可添加的专业技能}
\begin{itemize}[parsep=0.5ex]
    \item 查阅过RISC-V ISA、debug标准、Vector扩展的文档。对RISC-V有一定的了解。
\end{itemize}

\section{\faCogs\ 可添加的项目经历}


\begin{spacing}{1.125}
    \datedsubsection{\textbf{基于 git 的在线版本控制和开发协作平台}}{}
  课程项目,合作实现了在网页上的 git 仓库访问、commit,diff 和历史版本查看、issue,pull request、用户信息查看、CI/CD 配置等功能。项目使用 vue3+springboot+postgresql 开发。个人在项目中承担主要开发人员的角色:负责主要功能实现、开发规范制定、技术选型、代码审计和任务分配;负责前端部分的开发,采用 typescript+vue3+pinia+element3+unocss等主流技术;使用 vue-router 导航守卫在路由切换时防止多次请求。使用 axios 拦截器进行错误提示和 404 页跳转。
  \end{spacing}

% \begin{onehalfspacing}
%   \datedsubsection{\textbf{一种c风格简易编程语言的编译器}}{}
%   课程项目。针对一种c风格简易编程语言,使用C++实现了编译器的前端。项目使用flex和bison进行词法分析和语法分析,并生成AST。项目随后利用AST生成中间代码表示(四地址码),进行不可达代码消除和限定条件下的复制传播优化和条件跳转简化。
%   项目介绍:针对一种 c 风格简易编程语言,使用 c++ 配合 flex 和 bison 实现了编译器中的词法分析、
% 语法分析、语义分析和中间代码生成几部分。
% 项目职责:
% • 构建符合 flex 和 bison 规范的规则和产生式,并用上述工具构建语法树。
% • 在语义分析中,使用节点中包含哈希表的树在不同作用域中维护符号表并进行错误检查;使用
% variant 代替 union 在 AST 节点中维护不同类型的数据;应用简单的 RTTI 思想记录 AST 节点类型。
% • 项目使用智能指针,移动语义等特性减少拷贝,防止内存泄漏。
% -----------
% 项目介绍:针对一种 c 风格简易编程语言,使用 c++ 配合 flex 和 bison 实现了编译器的前端。
%在语义分析中,使用节点中包含哈希表的树在不同作用域中维护符号表并进行错误检查;使用
% variant 代替 union 在 AST 节点中维护不同类型的数据;应用简单的 RTTI 思想记录 AST 节点类型。
% • 项目使用智能指针,移动语义等特性减少拷贝,防止内存泄漏。

% \end{onehalfspacing}

% \begin{onehalfspacing}
%   \datedsubsection{\textbf{其他RISC-V相关项目}}{}
%   在某比赛中使用RISC-V Vector扩展加速GMSSL运算。在学校项目中使用RISC-V debug的Trigger配合GDB stub实现GDB远程调试裸机程序。
% \end{onehalfspacing}

\section{\faHeartO 可添加的获奖情况}
\begin{spacing}{1.125}
\datedsubsection{RoboMaster 2021 机甲大师超级对抗赛全国赛,\textbf{一等奖}}{2021 年 7 月}
在队伍中主要负责嵌入式系统部分。负责机器人的基础功能实现、调试,并利用 ttl 串口实现单片
机和运行在 linux 系统下的 c++ 程序通讯。
\end{spacing}
\begin{spacing}{1.125}
    \datedsubsection{第六届“强网杯”全国网络安全挑战赛线上赛,\textbf{强网先锋}}{2022 年 7 月
    }
\end{spacing}

\end{document}



\end{document}
